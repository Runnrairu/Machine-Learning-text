\documentclass{jsarticle}
\usepackage[dvipdfmx]{graphicx}
\usepackage{amsmath, amssymb}
\usepackage{type1cm}
\usepackage{bm}
\usepackage{here}
\usepackage{mathrsfs}
\usepackage[margin=2cm]{geometry}
\usepackage{wrapfig}
\usepackage[dvipdfmx]{hyperref}
\usepackage{pxjahyper}
\renewcommand{\refname}{参考文献}
\newtheorem{theo}{定理}
\newtheorem{ho}{補題}
\newtheorem{defi}{定義}

\begin{document}

\begin{center}
  
  \Huge Github版 \par
  \vspace{15mm}
  \Huge 一般的\footnote{ここで言う「一般的な」とは、「普通の人にもわかりやすい」という意味ではなく、数学界隈における「汎用性の高い」「広く対応可能な定義を導入している」という意味である。}な機械学習入門\par
  \vspace{90mm}
  \Large 最終更新:2020年10月11日 \par
  \Large 逢空れい@ranoiaru\par

\end{center}
\thispagestyle{empty}
\clearpage
\addtocounter{page}{-1}







\newpage


 \tableofcontents
 \clearpage
\section*{notation}
\begin{itemize}
\item $(\Omega,\mathcal{F},P)$:確率空間
\item $H$:可分ヒルベルト空間
\item $L$:損失関数$H\to \mathbb{R}$

\end{itemize}


\newpage
\section{はじめに}

深夜テンションによる出来心で「数学徒向けの機械学習入門書を書いたら見てくれる人ふぁぼください。ふぁぼ多かったら実行します」と書いたら、なんか一晩で500以上ふぁぼられたうえ、なぜか200近くフォロワーが増えたのでやらざるを得ない。

\subsection{本書執筆の経緯(よまなくてもいいよ)}
本書の構想は三年前に遡る。当時私は機械学習に入門したばかりの修士課程学生だったが、その時に触れた書籍があまりに数学的厳密性にも一般性にも欠け、そのことに対する苛立ちをツイートにぶつけた。

その結果、燃えた。

クソリプと暴言の嵐で通知が埋まり、私はしばらく鍵垢に籠った。

私の言い方も大いに悪かったのだが、あまりにも曲解されすぎではないかと当時の私は思ったものである。あの発言の意図としては、インターン時のパワハラ面接官や、twitterのひたすらマウントをとってくる某エンジニアに対して「数学ろくに知らないくせに『数学なんて機械学習に無意味』なんていうんじゃない」ということであり、断じて「純粋数学に明るくなきゃ本当に機械学習わかってるとは言えない」などということではない。誓ってもいい。

その時私は、数学徒向けの機械学習入門理論を整備すると宣言し、実際に修論にはそれを書いたり、数学徒のつどいで講演したりしたのだが、このように資料の形での公開はほとんどしていない。

機械学習界隈を騒がせてからちょうど三年。私も学生時代が終わり、数学寄りの機械学習研究者として仕事をしながら、修士時代の研究室へ社会人博士への入学をもくろむ日々。ツイートがバズったのもなにかの縁であろう。変な書き方のせいで不快な思いをさせてしまった方々へのお詫びも兼ねて、ここはひとつ、数学徒向け機械学習入門書の執筆を行うことにした。

\subsection{前提知識}
本資料において、入門編の前提知識は「学部レベルの解析学」(主に関数解析とルベーグ積分)、発展編はその都度必要な知識を補完されたし。


\newpage

\section{入門:機械学習の一般的問題設定}

\newpage
\begin{thebibliography}{99}
  \bibitem{キー} 
\end{thebibliography}

\end{document}
